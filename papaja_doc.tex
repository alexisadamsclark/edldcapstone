% Options for packages loaded elsewhere
\PassOptionsToPackage{unicode}{hyperref}
\PassOptionsToPackage{hyphens}{url}
%
\documentclass[
  english,
  man,draftall]{apa6}
\usepackage{lmodern}
\usepackage{amssymb,amsmath}
\usepackage{ifxetex,ifluatex}
\ifnum 0\ifxetex 1\fi\ifluatex 1\fi=0 % if pdftex
  \usepackage[T1]{fontenc}
  \usepackage[utf8]{inputenc}
  \usepackage{textcomp} % provide euro and other symbols
\else % if luatex or xetex
  \usepackage{unicode-math}
  \defaultfontfeatures{Scale=MatchLowercase}
  \defaultfontfeatures[\rmfamily]{Ligatures=TeX,Scale=1}
\fi
% Use upquote if available, for straight quotes in verbatim environments
\IfFileExists{upquote.sty}{\usepackage{upquote}}{}
\IfFileExists{microtype.sty}{% use microtype if available
  \usepackage[]{microtype}
  \UseMicrotypeSet[protrusion]{basicmath} % disable protrusion for tt fonts
}{}
\makeatletter
\@ifundefined{KOMAClassName}{% if non-KOMA class
  \IfFileExists{parskip.sty}{%
    \usepackage{parskip}
  }{% else
    \setlength{\parindent}{0pt}
    \setlength{\parskip}{6pt plus 2pt minus 1pt}}
}{% if KOMA class
  \KOMAoptions{parskip=half}}
\makeatother
\usepackage{xcolor}
\IfFileExists{xurl.sty}{\usepackage{xurl}}{} % add URL line breaks if available
\IfFileExists{bookmark.sty}{\usepackage{bookmark}}{\usepackage{hyperref}}
\hypersetup{
  pdftitle={COVID-19-Related Institutional Betrayal Among A Sample of Undergraduate Students},
  pdfauthor={Alexis Adams-Clark1,2 \& Jennifer Freyd1,2},
  pdflang={en-EN},
  pdfkeywords={institutional betrayal, institutional courage, trauma symptoms, COVID-19},
  hidelinks,
  pdfcreator={LaTeX via pandoc}}
\urlstyle{same} % disable monospaced font for URLs
\usepackage{graphicx,grffile}
\makeatletter
\def\maxwidth{\ifdim\Gin@nat@width>\linewidth\linewidth\else\Gin@nat@width\fi}
\def\maxheight{\ifdim\Gin@nat@height>\textheight\textheight\else\Gin@nat@height\fi}
\makeatother
% Scale images if necessary, so that they will not overflow the page
% margins by default, and it is still possible to overwrite the defaults
% using explicit options in \includegraphics[width, height, ...]{}
\setkeys{Gin}{width=\maxwidth,height=\maxheight,keepaspectratio}
% Set default figure placement to htbp
\makeatletter
\def\fps@figure{htbp}
\makeatother
\setlength{\emergencystretch}{3em} % prevent overfull lines
\providecommand{\tightlist}{%
  \setlength{\itemsep}{0pt}\setlength{\parskip}{0pt}}
\setcounter{secnumdepth}{-\maxdimen} % remove section numbering
% Make \paragraph and \subparagraph free-standing
\ifx\paragraph\undefined\else
  \let\oldparagraph\paragraph
  \renewcommand{\paragraph}[1]{\oldparagraph{#1}\mbox{}}
\fi
\ifx\subparagraph\undefined\else
  \let\oldsubparagraph\subparagraph
  \renewcommand{\subparagraph}[1]{\oldsubparagraph{#1}\mbox{}}
\fi
% Manuscript styling
\usepackage{upgreek}
\captionsetup{font=singlespacing,justification=justified}

% Table formatting
\usepackage{longtable}
\usepackage{lscape}
% \usepackage[counterclockwise]{rotating}   % Landscape page setup for large tables
\usepackage{multirow}		% Table styling
\usepackage{tabularx}		% Control Column width
\usepackage[flushleft]{threeparttable}	% Allows for three part tables with a specified notes section
\usepackage{threeparttablex}            % Lets threeparttable work with longtable

% Create new environments so endfloat can handle them
% \newenvironment{ltable}
%   {\begin{landscape}\begin{center}\begin{threeparttable}}
%   {\end{threeparttable}\end{center}\end{landscape}}
\newenvironment{lltable}{\begin{landscape}\begin{center}\begin{ThreePartTable}}{\end{ThreePartTable}\end{center}\end{landscape}}

% Enables adjusting longtable caption width to table width
% Solution found at http://golatex.de/longtable-mit-caption-so-breit-wie-die-tabelle-t15767.html
\makeatletter
\newcommand\LastLTentrywidth{1em}
\newlength\longtablewidth
\setlength{\longtablewidth}{1in}
\newcommand{\getlongtablewidth}{\begingroup \ifcsname LT@\roman{LT@tables}\endcsname \global\longtablewidth=0pt \renewcommand{\LT@entry}[2]{\global\advance\longtablewidth by ##2\relax\gdef\LastLTentrywidth{##2}}\@nameuse{LT@\roman{LT@tables}} \fi \endgroup}

% \setlength{\parindent}{0.5in}
% \setlength{\parskip}{0pt plus 0pt minus 0pt}

% \usepackage{etoolbox}
\makeatletter
\patchcmd{\HyOrg@maketitle}
  {\section{\normalfont\normalsize\abstractname}}
  {\section*{\normalfont\normalsize\abstractname}}
  {}{\typeout{Failed to patch abstract.}}
\patchcmd{\HyOrg@maketitle}
  {\section{\protect\normalfont{\@title}}}
  {\section*{\protect\normalfont{\@title}}}
  {}{\typeout{Failed to patch title.}}
\makeatother
\shorttitle{COVID-19 Institutional Betrayal}
\keywords{institutional betrayal, institutional courage, trauma symptoms, COVID-19\newline\indent Word count: X}
\DeclareDelayedFloatFlavor{ThreePartTable}{table}
\DeclareDelayedFloatFlavor{lltable}{table}
\DeclareDelayedFloatFlavor*{longtable}{table}
\makeatletter
\renewcommand{\efloat@iwrite}[1]{\immediate\expandafter\protected@write\csname efloat@post#1\endcsname{}}
\makeatother
\usepackage{lineno}

\linenumbers
\usepackage{csquotes}
\ifxetex
  % Load polyglossia as late as possible: uses bidi with RTL langages (e.g. Hebrew, Arabic)
  \usepackage{polyglossia}
  \setmainlanguage[]{english}
\else
  \usepackage[shorthands=off,main=english]{babel}
\fi

\title{COVID-19-Related Institutional Betrayal Among A Sample of Undergraduate Students}
\author{Alexis Adams-Clark\textsuperscript{1,2} \& Jennifer Freyd\textsuperscript{1,2}}
\date{}


\authornote{

University of Oregon, Department of Psychology, 1227 University St.~Eugene, OR 97401
Center for Institutional Courage, Inc., Palo Alto, CA

The authors made the following contributions. Alexis Adams-Clark: Conceptualization, Writing - Original Draft Preparation, Writing - Review \& Editing; Jennifer Freyd: Conceptualization, Writing - Review \& Editing.

Correspondence concerning this article should be addressed to Alexis Adams-Clark, 1227 University St.~Eugene, OR 97401. E-mail: \href{mailto:aadamscl@uoregon.edu}{\nolinkurl{aadamscl@uoregon.edu}}

}

\affiliation{\vspace{0.5cm}\textsuperscript{1} University of Oregon\\\textsuperscript{2} Center for Institutional Courage}

\abstract{
One or two sentences providing a \textbf{basic introduction} to the field, comprehensible to a scientist in any discipline.

Two to three sentences of \textbf{more detailed background}, comprehensible to scientists in related disciplines.

One sentence clearly stating the \textbf{general problem} being addressed by this particular study.

One sentence summarizing the main result (with the words ``\textbf{here we show}'' or their equivalent).

Two or three sentences explaining what the \textbf{main result} reveals in direct comparison to what was thought to be the case previously, or how the main result adds to previous knowledge.

One or two sentences to put the results into a more \textbf{general context}.

Two or three sentences to provide a \textbf{broader perspective}, readily comprehensible to a scientist in any discipline.
}



\begin{document}
\maketitle

\hypertarget{methods}{%
\section{Methods}\label{methods}}

\hypertarget{participants}{%
\subsection{Participants}\label{participants}}

Participants were recruited from the Human Subjects Pool at a large, public university in the Northwest United States. The university's Human Subjects Pool contains undergraduate students currently enrolled in introductory psychology and linguistics courses, and these students receive course credit for their participation in research studies. Students are not aware of the topic of any given study prior to signing up, which reduces self-selection bias (although they do have the option to end participation during the informed consent process or at any time throughout the study). A total of 346 undergraduate students signed up and consented to participate in the present study. Participants who failed to correctly answer at least four out of five \enquote{attention check} questions randomly located throughout the survey (e.g.~"please choose \enquote{strongly agree} if you are paying attention) were removed prior to data analysis (n = 37). The final sample used for data analysis consisted of 309 participants (71.5\% women, 26.5\% men, 1.9\% non-binary/gender-nonconforming). The majority of participants were White (63.4\%), and the average age of participants was 19.39 (SD = 1.45).

These data were collected during the fall 2020 term of the academic year, during which COVID-19 infections were steadily climbing at the university, local, and national level. The university at the focus of the current investigation adopted a largely remote learning environment. However, the university required all first-year students to live in dormitories on campus, and a minority of classes were held in person.

\hypertarget{method}{%
\subsection{Method}\label{method}}

\textbf{COVID-19-Related Institutional Betrayal}. COVID-19-related institutional betrayal was measured using an adapted version of the Institutional Betrayal Questionnaire (IBQ; Smith and Freyd (2013), Smith and Freyd (2017)). The IBQ consists of 12 items listing actions or inactions by an institution in rsponse to a traumatic event, and it has been established as a valid measure of institutional betrayal. Although originally designed to assess the university's responses to instances of sexual violence, the measure was adapted to apply to the university's responses to the COVID-19 pandemic. Participants were instructed to answer each item (e.g., \enquote{did your university create an environment in which COVID-19 infection and safety violations seemed more likely?}) by selecting \enquote{Yes,} \enquote{No,} or \enquote{Not Applicable.} \enquote{Yes} responses were coded as 1 and were summed to create a total IBQ score ranging from 0 to 12. The distribution was skewed (1.38) and kurtotic (1.31), but within the range in which the assumption of normality can be maintained without transformation. At the end of the scale, students were asked to rate the extent to which they identified with the institution both prior to and since the COVID-19 pandemic.

\textbf{Trauma-related symptoms.} General trauma-related symptoms were measured using the Trauma Symptoms Checklist (Elliott and Briere (1992)), which is a valid, standard measure of various symptoms that may be related to traumatic experiences. The scale consists of several subscales, including Dissociation subscale, Sleep Disturbance subscale, Sexual Problems subscale, Anxiety subscale, Depression subscale, and the Sexual Abuse Trauma Index subscale. For the present study, only the total overall TSC score was used for analysis, and items were summed and averaged to create an \enquote{average} TSC score for each participant. Participants were asked to rate the frequency of various symptoms in the past two months, using the anchors ranging from 0 (\enquote{Never}) to 3 (\enquote{Often}). The scale demonstrated satisfactory reliability in this current study (alpha = .94).

\textbf{COVID-19-specific trauma cognitions.} COVID-19-specific trauma cognitions were measured using an adapted version of the Impact of Events Scale (IES; Horowitz, Wilner, and Alvarez (1979)) that has been adapted to COVID-19 (ref). This scale specifically measures avoidance and intrusion cognitions related to COVID-19. Avoidance and intrusive symptoms often occur following exposure to a specific traumatic event. Participants were asked to rate the frequency of each symptom (e.g., \enquote{I had trouble falling asleep because thoughts about COVID-19 came into my mind}) in the past week, using anchors ranging from 0 (\enquote{Never}) to 3 (\enquote{Often}). Items were summed and averaged to create an \enquote{average} IES score for each participant. The scale demonstrated satisfactory reliability in this current study (alpha = .90).

\hypertarget{procedure}{%
\subsection{Procedure}\label{procedure}}

In the present study, all participants reviewed an informed consent form before participation. After consenting to participate, participants completed questionnaires through an online survey hosted on \emph{Qualtrics} from a personal electronic device in a private location of their choosing. Participants had the option to leave items blank and to discontinue participation at any time. Upon completion of the survey, participants reviewed a debriefing form and received course credit for their participation. All study procedures were approved by the university's Office of Research Compliance (Institutional Review Board).

\hypertarget{data-analysis}{%
\subsection{Data analysis}\label{data-analysis}}

We used R (Version 4.0.2; R Core Team, 2020) and the R-packages \emph{apaTables} (Version 2.0.5; Stanley, 2018), \emph{corx} (Version 1.0.6.1; Conigrave, 2020), \emph{dplyr} (Version 1.0.4; Wickham, François, Henry, \& Müller, 2021), \emph{forcats} (Version 0.5.0; Wickham, 2020a), \emph{ggplot2} (Version 3.3.2; Wickham, 2016), \emph{here} (Version 0.1; Müller, 2017), \emph{papaja} (Version 0.1.0.9997; Aust \& Barth, 2020), \emph{psych} (Version 2.0.7; Revelle, 2020), \emph{purrr} (Version 0.3.4; Henry \& Wickham, 2020), \emph{readr} (Version 1.3.1; Wickham, Hester, \& Francois, 2018), \emph{rio} (Version 0.5.16; Chan, Chan, Leeper, \& Becker, 2018), \emph{skimr} (Version 2.1.2; Waring et al., 2020), \emph{stringr} (Version 1.4.0; Wickham, 2019), \emph{tibble} (Version 3.0.6; Müller \& Wickham, 2021), \emph{tidyr} (Version 1.1.2; Wickham, 2020b), and \emph{tidyverse} (Version 1.3.0; Wickham et al., 2019) for all our analyses.

\hypertarget{results}{%
\section{Results}\label{results}}

\begin{figure}[H]

{\centering \includegraphics[width=\textwidth]{papaja_doc_files/figure-latex/figure1-1} 

}

\caption{Percentage of Students Endorsing Institutional Betrayal
}\label{fig:figure1}
\end{figure}

\begin{figure}[H]

{\centering \includegraphics[width=\textwidth]{papaja_doc_files/figure-latex/figure2-1} 

}

\caption{Institutional Betrayal Score by Gender (N = 309) 
}\label{fig:figure2}
\end{figure}

The majority of students (66.34\%) reported at least one type of COVID-19-related institutional betrayal. The most common types of institutional betrayal reported were \enquote{creating an environment in which COVID-19 transmission was more common or seemed normal} and \enquote{failure to prevent COVID-19 transmission} (See Figure \ref{fig:figure1}). There were no significant differences in COVID-19-related institutional betrayal by gender (See Figure \ref{fig:figure2}).

\begin{table}[tbp]

\begin{center}
\begin{threeparttable}

\caption{\label{tab:table1}Example corr matrix}

\begin{tabular}{lllll}
\toprule
 & \multicolumn{1}{c}{1} & \multicolumn{1}{c}{2} & \multicolumn{1}{c}{$M$} & \multicolumn{1}{c}{$SD$}\\
\midrule
1. Institutional Betrayal Score & - &  & 2.47 & 2.90\\
2. Trauma Symptom Score & .22*** & - & 0.87 & 0.53\\
3. Impact of Event Score & .21*** & .44*** & 1.07 & 0.64\\
\bottomrule
\addlinespace
\end{tabular}

\begin{tablenotes}[para]
\normalsize{\textit{Note.} * p < 0.05; ** p < 0.01; *** p < 0.001}
\end{tablenotes}

\end{threeparttable}
\end{center}

\end{table}

\begin{table}[tbp]

\begin{center}
\begin{threeparttable}

\caption{\label{tab:table2}A full regression table.}

\begin{tabular}{lllll}
\toprule
Predictor & \multicolumn{1}{c}{$b$} & \multicolumn{1}{c}{95\% CI} & \multicolumn{1}{c}{$t(276)$} & \multicolumn{1}{c}{$p$}\\
\midrule
Intercept & 0.33 & $[0.16$, $0.50]$ & 3.78 & < .001\\
GenderWoman & 0.19 & $[0.04$, $0.34]$ & 2.54 & .012\\
GenderTrans/Non-conforming/Non-binary & 0.33 & $[-0.13$, $0.79]$ & 1.43 & .154\\
Covid19know someone with covid & 0.21 & $[0.07$, $0.35]$ & 3.00 & .003\\
Tsc mean & 0.40 & $[0.27$, $0.53]$ & 6.18 & < .001\\
Ibq sum & 0.03 & $[0.00$, $0.05]$ & 2.39 & .017\\
\bottomrule
\addlinespace
\end{tabular}

\begin{tablenotes}[para]
\normalsize{\textit{Note.} * p < 0.05; ** p < 0.01; *** p < 0.001}
\end{tablenotes}

\end{threeparttable}
\end{center}

\end{table}

Institutional betrayal was significantly associated with both general trauma-related symptoms and COVID-19 specific avoidance and intrusion symptoms, \emph{p} \textless{} .001 (see Table \ref{tab:table1}). Institutional betrayal was associated with unique variance in COVID-19 specific avoidance and intrusion symptoms, \emph{p} = .01 (see Table \ref{tab:table2}), even when controlling for gender, knowing someone close with COVID-19, and non-specific trauma-related distress.

\begin{figure}[H]

{\centering \includegraphics[width=\textwidth]{papaja_doc_files/figure-latex/figure3-1} 

}

\caption{Institutional Identity by Institutional Betrayal (N = 309) 
}\label{fig:figure3}
\end{figure}

\begin{table}[tbp]

\begin{center}
\begin{threeparttable}

\caption{\label{tab:table3}A full regression table.}

\begin{tabular}{lllll}
\toprule
Predictor & \multicolumn{1}{c}{$b$} & \multicolumn{1}{c}{95\% CI} & \multicolumn{1}{c}{$t(301)$} & \multicolumn{1}{c}{$p$}\\
\midrule
Intercept & 0.07 & $[-0.08$, $0.21]$ & 0.89 & .377\\
Scaleid 1before & 0.21 & $[0.06$, $0.35]$ & 2.79 & .006\\
Ibq sum & -0.02 & $[-0.06$, $0.02]$ & -0.86 & .390\\
Scaleid 1before $\times$ Ibq sum & -0.06 & $[-0.10$, $-0.02]$ & -2.86 & .005\\
\bottomrule
\addlinespace
\end{tabular}

\begin{tablenotes}[para]
\normalsize{\textit{Note.} * p < 0.05; ** p < 0.01; *** p < 0.001}
\end{tablenotes}

\end{threeparttable}
\end{center}

\end{table}

\hypertarget{discussion}{%
\section{Discussion}\label{discussion}}

\newpage

\hypertarget{references}{%
\section{References}\label{references}}

\begingroup
\setlength{\parindent}{-0.5in}
\setlength{\leftskip}{0.5in}

\hypertarget{refs}{}
\leavevmode\hypertarget{ref-R-papaja}{}%
Aust, F., \& Barth, M. (2020). \emph{papaja: Create APA manuscripts with R Markdown}. Retrieved from \url{https://github.com/crsh/papaja}

\leavevmode\hypertarget{ref-R-rio}{}%
Chan, C.-h., Chan, G. C., Leeper, T. J., \& Becker, J. (2018). \emph{Rio: A swiss-army knife for data file i/o}.

\leavevmode\hypertarget{ref-R-corx}{}%
Conigrave, J. (2020). \emph{Corx: Create and format correlation matrices}. Retrieved from \url{https://CRAN.R-project.org/package=corx}

\leavevmode\hypertarget{ref-elliott1992}{}%
Elliott, D. M., \& Briere, J. (1992). Sexual abuse trauma among professional women: Validating the trauma symptom checklist-40 (tsc-40). \emph{Child Abuse \& Neglect}, \emph{16}(3), 391--398.

\leavevmode\hypertarget{ref-R-purrr}{}%
Henry, L., \& Wickham, H. (2020). \emph{Purrr: Functional programming tools}. Retrieved from \url{https://CRAN.R-project.org/package=purrr}

\leavevmode\hypertarget{ref-horowitz1979}{}%
Horowitz, M., Wilner, N., \& Alvarez, W. (1979). Impact of event scale: A measure of subjective stress. \emph{Psychosomatic Medicine}, \emph{41}(3), 209--218.

\leavevmode\hypertarget{ref-R-here}{}%
Müller, K. (2017). \emph{Here: A simpler way to find your files}. Retrieved from \url{https://CRAN.R-project.org/package=here}

\leavevmode\hypertarget{ref-R-tibble}{}%
Müller, K., \& Wickham, H. (2021). \emph{Tibble: Simple data frames}. Retrieved from \url{https://CRAN.R-project.org/package=tibble}

\leavevmode\hypertarget{ref-R-base}{}%
R Core Team. (2020). \emph{R: A language and environment for statistical computing}. Vienna, Austria: R Foundation for Statistical Computing. Retrieved from \url{https://www.R-project.org/}

\leavevmode\hypertarget{ref-R-psych}{}%
Revelle, W. (2020). \emph{Psych: Procedures for psychological, psychometric, and personality research}. Evanston, Illinois: Northwestern University. Retrieved from \url{https://CRAN.R-project.org/package=psych}

\leavevmode\hypertarget{ref-smith2013}{}%
Smith, C. P., \& Freyd, J. J. (2013). Dangerous safe havens: Institutional betrayal exacerbates sexual trauma. \emph{Journal of Traumatic Stress}, \emph{26}(1), 119--124.

\leavevmode\hypertarget{ref-smith2017}{}%
Smith, C. P., \& Freyd, J. J. (2017). Insult, then injury: Interpersonal and institutional betrayal linked to health and dissociation. \emph{Journal of Aggression, Maltreatment \& Trauma}, \emph{26}(10), 1117--1131.

\leavevmode\hypertarget{ref-R-apaTables}{}%
Stanley, D. (2018). \emph{ApaTables: Create american psychological association (apa) style tables}. Retrieved from \url{https://CRAN.R-project.org/package=apaTables}

\leavevmode\hypertarget{ref-R-skimr}{}%
Waring, E., Quinn, M., McNamara, A., Arino de la Rubia, E., Zhu, H., \& Ellis, S. (2020). \emph{Skimr: Compact and flexible summaries of data}. Retrieved from \url{https://CRAN.R-project.org/package=skimr}

\leavevmode\hypertarget{ref-R-ggplot2}{}%
Wickham, H. (2016). \emph{Ggplot2: Elegant graphics for data analysis}. Springer-Verlag New York. Retrieved from \url{https://ggplot2.tidyverse.org}

\leavevmode\hypertarget{ref-R-stringr}{}%
Wickham, H. (2019). \emph{Stringr: Simple, consistent wrappers for common string operations}. Retrieved from \url{https://CRAN.R-project.org/package=stringr}

\leavevmode\hypertarget{ref-R-forcats}{}%
Wickham, H. (2020a). \emph{Forcats: Tools for working with categorical variables (factors)}. Retrieved from \url{https://CRAN.R-project.org/package=forcats}

\leavevmode\hypertarget{ref-R-tidyr}{}%
Wickham, H. (2020b). \emph{Tidyr: Tidy messy data}. Retrieved from \url{https://CRAN.R-project.org/package=tidyr}

\leavevmode\hypertarget{ref-R-tidyverse}{}%
Wickham, H., Averick, M., Bryan, J., Chang, W., McGowan, L. D., François, R., \ldots{} Yutani, H. (2019). Welcome to the tidyverse. \emph{Journal of Open Source Software}, \emph{4}(43), 1686. \url{https://doi.org/10.21105/joss.01686}

\leavevmode\hypertarget{ref-R-dplyr}{}%
Wickham, H., François, R., Henry, L., \& Müller, K. (2021). \emph{Dplyr: A grammar of data manipulation}. Retrieved from \url{https://CRAN.R-project.org/package=dplyr}

\leavevmode\hypertarget{ref-R-readr}{}%
Wickham, H., Hester, J., \& Francois, R. (2018). \emph{Readr: Read rectangular text data}. Retrieved from \url{https://CRAN.R-project.org/package=readr}

\endgroup


\end{document}
